\chapter{Discussion}
\label{chap:discussion}

We have described a broad spectrum of experiments in quantum photonics, many of which make use of the control and complexity afforded by monolithic integration. In our work with the \gls{cnotmz}, we have shown the value of reconfigurability in such devices, and the surprising diversity of experiments which can be performed with just two qubits. In doing so, we have confirmed that integrated quantum photonics can reproduce the performance and flexibility of bulk optics. 

Using this device we have implemented a new variant on Wheeler's delayed choice experiment, observing continuous tuning between wave and particle phenomena for the first time. While we do not contend that this result provides new physical understanding over and above Bell's theorem, for example, we suggest that it nonetheless provides a useful pedagogical tool to think about wave-particle duality.

In chapter \ref{chap:random-chsh}, we introduced three new protocols, which allow the presence of entanglement to be certified under suboptimal experimental conditions. It is reasonable to think that these techniques will be useful for the characterization of quantum states in the laboratory, where calibration and alignment can sometimes be problematic. We believe that these methods might also find applications in quantum key distribution and related quantum communication protocols, when two distant parties do not share a common frame.

Chapter \ref{chap:quantum-chemistry} introduced a new algorithm for quantum chemistry. Although the analysis is not complete, we believe that this technique potentially offers very significant benefits over the current \emph{status quo} for quantum simulation, particularly with respect to the number of gate operations required. Even if this algorithm is not used in the exact form described here, we anticipate that realistic implementations of quantum simulators will need to adopt the pragmatic approach described here. We ran our algorithm using the \gls{cnotmz}, demonstrating both the ability of the algorithm to simulate larger systems with fewer resources, as well as further testing the performance and repeatability of the integrated quantum chip.

Finally, chapter \ref{chap:hilbert-space-telescope} describes a number of technical advances in both state preparation and measurement. As with the \gls{cnotmz}, we again see that by increasing the number of experimental degrees of freedom by a relatively small amount, we expand the diversity and power of experimental quantum phenomena very significantly. We expect that successful verification techniques for \bosonsampling-like problems, if not following exactly the method outlined in chapter \ref{chap:hilbert-space-telescope}, will at least depend on the fundamental ideas described therein: namely, deliberate introduction and exploitation of structure in the device and resulting probability distribution.


