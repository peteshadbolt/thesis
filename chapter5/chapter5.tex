
\chapter{Entanglement and nonlocality without a shared frame}
\label{chap:random-chsh}

\section{Introduction}
\label{sec:random-chsh-intro}

In many quantum information tasks, the basic scenario is one of two parties, Alice and Bob, who share an entangled state $\ket{\psi_{AB}}$ originating from a source. Alice and Bob may wish to use this state to communicate securely (section \ref{sec:qkd}), violate a Bell inequality (\ref{sec:nonlocality}), perform teleportation, tomography (\ref{sec:state-tomography}), or to evaluate the degree of entanglement of the state (\ref{sec:entanglement}). Perhaps they are space-like separated, maybe they are in the same lab, perhaps $\ket{\psi_{AB}}$ is a resource state in a quantum computer --- we have already discussed many such scenarios.

One assumption that is very often made in theoretical works is that Alice and Bob share a reference frame. That is, they agree on a coordinate system in which Bob's ``up'' is the same as Alice's, and they can, for example, measure qubits in the $\pauli_{x,y,z}$ bases. This assumption is often valid --- in proof-of-principle experiments we usually operate within the frame of the laboratory, and have classical tools at our disposal to precisely calibrate and align Alice and Bob with respect to one another. However, there are many real-world scenarios in which full calibration and alignment is not possible.

In single-mode optical fiber, natural and unavoidable fluctuations in temperature and stress give rise to unknown, random, unitary rotations of the polarization of transmitted light \cite{Gisin2002}. 
These rotations roughly span the entire space of $SU(2)$ --- although not in any uniform way --- and largely preclude the use of polarization encoding in classical telecommunications. 
Optical satellite links, proposed as a real-world target for photonic quantum communication \cite{Wang2013, Ursin2009, Zhang2013},  suffer from continuous rotation of the satellite with respect to earth, as well as timing drift, necessitating complex tracking and correction systems 
(figure \ref{fig:random-chsh-setup}).
Path encoded qubits in bulk suffer from thermal/acoustic phase instability, which gives rise to unknown random unitary rotations of the qubit reference frame.
Even if the setup is perfectly stable, we sometimes just do not have the time or tools to calibrate phaseshifters, waveplates, and polarization controllers. 
As quantum technologies become increasingly complex, these issues will not disappear.

In all such scenarios, unknown rotations decouple Alice's reference frame from Bob, effectively \emph{breaking} most tomographic protocols, entanglement witnesses, Bell tests, and QKD. Sometimes we can use active stabilization or further classical communication to establish a shared frame, but it is interesting to ask --- how well can we perform these QI tasks in the absence of any shared reference frame?

In this chapter we show how detection of Bell nonlocality can be guaranteed --- preserving device independence --- without a shared frame, even in the absence of well-calibrated devices. We experimentally demonstrate that by randomizing voltages on the \acrshort{cnotmz}, we can violate a Bell inequality with high probability.  Finally, we describe a practical method to accurately measure the degree of entanglement of a two-qubit state despite time-dependent unitary noise on the local channel between source and observer. This method makes direct use of Haar-random noise to improve performance,  and allows an experimentalist to detect entanglement by simply shaking, bending and twisting non-polarization maintaining optical fiber. We discuss possible applications of this scheme to measurement and secure communication.

\section{Bell tests without a shared frame} 
\label{sec:random-chsh-calibrated}

\begin{figure}[!t]
\centering
\includegraphics[width=1\linewidth]{chapter5/fig/chsh/setup.pdf}
\caption[Experimental setup]{
\label{fig:random-chsh-setup}  
Bell violations with random measurements. (a)
A source generates entangled pairs, and photons are sent to Alice and Bob respectively. We consider a scenario in which Alice and Bob do not share a frame of reference --- that is, they cannot choose a common measurement basis --- and are therefore forced to measure in randomly oriented bases. (b) We use the \acrshort{cnotmz} to experimentally test a scheme which guarantees Bell inequality violation even in the absence of a shared reference frame. Path-entangled photon pairs are generated by the \acrshort{cnotp} gate and measured in a qubit basis by Alice and Bob, who are implemented using the readout stage of the \acrshort{cnotmz}. The choice of measurement setting is accomplished using thermal phase shifters $\phi_{5-8}$. 
} 
\end{figure}

%FUTURE: this really is crap
In sections \ref{sec:nonlocality}, \ref{sec:qkd}, and \ref{sec:cnot-mz-chsh} of this thesis we have seen the significance of nonlocality as a fundamental quantum mechanical phenomenon, as well the utility of nonlocal correlations as a tool for device-independent quantum communication and state characterization.
Bell tests such as Bell-\acrshort{chsh}, described in detail in section \ref{sec:nonlocality}, provide an experimental prescription for rigorous certification of nonlocal statistics. 

It will be convenient to first re-write the Bell-CHSH inequality (\ref{eqn:chsh-abstract}) using a slightly different notation.  We assume that Alice and Bob measure a two-qubit state $\ket{\psi}$ using $m$ local measurement settings per party, $\hat{A}_j, \hat{B}_j$, $i, j \in \left[0, m -1 \right]$ respectively.  For $m=2$, all local hidden variable (LHV) models must satisfy the Bell-CHSH inequality
\begin{equation}
        |S| = |\expect{\hat{A}_0 \hat{B}_0} + 
        \expect{\hat{A}_0 \hat{B}_1} + 
        \expect{\hat{A}_1 \hat{B}_0} - 
        \expect{\hat{A}_1 \hat{B}_1}|
        \le 2.
\label{eqn:chsh-two}
\end{equation}
Since the indexing of each measurement setting $\hat{A}_i, \hat{B}_j$ is arbitrary, terms in \ref{eqn:chsh-abstract} can be possibly permuted, moving the minus sign and creating a number of equally valid Bell inequalities. Local-realistic models satisfy all such permutations. 
For instance, 
       $|\expect{\hat{A}_1 \hat{B}_0} + 
        \expect{\hat{A}_1 \hat{B}_1} + 
        \expect{\hat{A}_0 \hat{B}_0} - 
        \expect{\hat{A}_0 \hat{B}_1}|
        \le 2$
holds for all LHV theories. Violation of any of the 36 allowed inequalities witnesses nonlocal behaviour.

For two qubits, although entanglement is necessary in order to obtain nonlocal statistics, it is not sufficient\footnote{Note that the picture is more complex for multi-particle scenarios, where nonlocality can be seen without entanglement. See for example \cite{Bennett1999b}}. Even if Alice and Bob share a maximally entangled state, they will not necessarily violate \gls{chsh} if they do not \emph{measure} in appropriate bases.  
To see this, first assume that Alice and Bob share the maximally entangled Bell state $\ket{\Psi^-}$. 
Letting 
$\hat{A}_0 = \pauli_x,~ \hat{A}_1 = \pauli_z,~ \hat{B}_0 = \pauli_x,~ \hat{B}_1 = \pauli_z$
, it is simple to show that $S = 0$, yielding no violation, and no nonlocal correlations. 
However, if we rotate $\hat{B}_j$,
\begin{equation}
\hat{B}_0 = \frac{ \pauli_x + \pauli_z }{\sqrt{2}},~~~~~~ \hat{B}_1 = \frac{ \pauli_x - \pauli_z }{ \sqrt{2}}   
\end{equation}
 we recover maximal violation of CHSH, $|S|=2\sqrt{2}$. In the theoretical discussion of such scenarios it is often implicitly assumed that Alice and Bob share a reference frame. How does \gls{chsh} perform when there is no common frame?

\subsection{Theory}
Let us assume that Alice and Bob share 
the singlet state $\ket{\Psi^-}$, but 
have no information that would allow them to establish a shared reference frame, and that they are interested in violating Bell-CHSH with the greatest possible efficiency. 
In this discussion it will be useful to consider the measurement settings of Alice and Bob in terms of their Bloch vectors (\ref{eqn:bloch-vectors}) $\vec{a}_i, ~\vec{b}_j \in \mathbb{R}^3$.
Alice and Bob each choose two vectors $\vec{a}_{0,1}$ and $\vec{b}_{0,1}$, independently from a uniform distribution over the 2-sphere (equivalent to the Haar measure for $SU(2)$, see section \ref{sec:quantum-mechanics-states}), and measure in all combinations of $\hat{A}_i \hat{B}_j$.  In 2010, Liang et al. showed \cite{Liang2010} that Bell violation is achieved in this scenario with a probability of $\sim 28\%$.  
If Alice and Bob are each able to choose \emph{mutually unbiased} vectors, orthonormal in the Bloch sphere and obeying $\vec{a}_i \cdot \vec{a}_j = \delta_{ij}$, this probability increases to $\sim 42 \%$. This analysis has been generalized to the multipartite case \cite{Liang2010, Wallman2011a}, as well as to schemes involving decoherence-free subspaces \cite{Cabello2003}. These results show that it is more probable to detect nonlocality than one might na\"ively expect. However, can it be guaranteed?

So far we have allowed only two measurement settings per party. 
Consider now a scenario in which each party chooses \emph{three} settings,
$\hat{A}_{0,1,2}$ and $\hat{B}_{0,1,2}$  where we again demand that these measurements are mutually unbiased, thus forming randomly-oriented orthogonal triads in the Bloch spheres of Alice and Bob respectively. It turns out that in this situation, we can \emph{always} find a valid Bell inequality of the form (\ref{eqn:chsh-two}) which is violated. In other words, by adding one measurement setting per party, detection of nonlocality can be guaranteed.

\vspace{5mm}
\noindent\textbf{Proof:}
Assume that $\vec{a}_{i\in \{0,1,2\}}$ and $\vec{b}_{i\in \{0,1,2\}}$ are mutually unbiased vectors corresponding to qubit measurement operators $\hat{A}_i$ and $\hat{B}_j$. 
Alice and Bob evaluate these expectation values for the singlet state $\ket{\Psi^-}$ over all combinations of $i, j$ and can then write them in matrix form,
\begin{eqnarray}
\label{matrix}
\mathcal{E} = \left(
\begin{array}{ccc}
E_{00} & E_{01} & E_{02} \\
E_{10} & E_{11} & E_{12} \\
E_{20} & E_{21} & E_{22}
\end{array}
\right).
\label{eqn:chsh-three-matrix}
\end{eqnarray}
It is straightforward to show that these expectation values are given by the scalar product $E_{ij} = \expect{\hat{A}_i \hat{B}_j} = -\vec{a}_i \cdot \vec{b}_j$.  
The columns of $\mathcal{E}$ are therefore equivalent to the coordinates $b_k'$ of the Bloch vectors $\vec{b}_j$ in the basis $\vec{a}_i\}$.  
By re-labelling measurement settings and outcomes we are free to permute rows and/or columns of this matrix as well as possibly change their sign. 
We can therefore assume, without loss of generality, that $E_{00, 11, 22} > 0$, and that $E_{22}$ is the largest element by absolute value in $\mathcal{E}$. 
$\vec{b}_j$ are orthonormal, giving 
$\vec{b}_2 = \pm \vec{b}_0 \times \vec{b}_1$ and therefore 
$|E_{22}| = |E_{00}E_{11} - E_{01}E_{10}|$.
Now, 
$E_{22} = E_{00} E_{11} - E_{01} E_{10}
\geq E_{00} , E_{11} , |E_{01}|, |E_{10}|$ 
and
$E_{01} E_{10} \leq 0$. 
We assume that $E_{01} \leq 0$ and $E_{10} \geq 0$, if this is not the case then we are free to multiply the second row and column by $-1$.  Now we have that
\begin{equation}
(E_{00} + E_{10}) \max[-E_{01},E_{11}] \geq E_{00} E_{11} - E_{01} E_{10} = E_{22} \geq \max[-E_{01},E_{11}].
\end{equation}
Dividing by $\max[-E_{01},E_{11}] > 0$, we find 
$E_{00} + E_{10} \geq 0$. Using a similar method, we can show 
$ -E_{01} + E_{11} \geq 0$.
Adding these inequalities, we obtain
\begin{equation}
E_{00} + E_{10} - E_{01} + E_{11} \geq 1 \,. \label{ineq_proof}
\end{equation}
By construction, $\mathcal{E}$ is an orthogonal matrix. Therefore, this inequality is satisfied if and only if 
$E_{00} + E_{10} = 0$, 
$-E_{01} + E_{11} = 0$ 
and 
$\vec{a}_0 = \vec{b}_0$, $\vec{a}_1 = \vec{b}_1$ and $\vec{a}_2 = \vec{b}_2$. That is, so long as Alice's measurements are not \emph{perfectly} aligned with respect to Bob's, \gls{chsh} is violated.
$\hfill\blacksquare$

An independent proof of this result was obtained by Wallman and Bartlett \cite{Wallman2012}, and published shortly after our manuscript appeared in \emph{Scientific Reports}.

\vspace{5mm}
We have shown that \gls{chsh} can be violated with certainty without a shared reference frame, when Alice and Bob share a perfect maximally entangled state. However, in order for this scheme to be practically relevant, we must consider its performance under realistic experimental imperfections. 

\begin{figure}[!t]
\includegraphics[width=\linewidth]{chapter5/fig/chsh/numerics.pdf}
\caption[Bell tests using random measurement triads.]{\label{fig:theory} (a) Bell tests using random measurement triads. Numerically computed distribution of maximum \gls{chsh} violation for
uniformly random, mutually unbiased measurement triads on a singlet state. (b) Bell tests using completely random measurements, without calibration. Numerical calculation of the probability of Bell violation as a function of Werner state visibility $V$, for different numbers $m$ of uniformly random random measurements per party.}
\label{fig:chsh-distribution}
\end{figure}

Experimental Bell tests are necessarily limited to measuring finite statistics, resulting in uncertainty in measured expectation values. This gives rise to error in $S$, and we must therefore examine the distribution of \gls{chsh} over all allowed measurement settings to ensure that the probability of violation remains high despite such uncertainty. Figure \ref{fig:chsh-distribution}(a) shows a numerical calculation of the distribution of $|S|$ when $\vec{a}_i$, $\vec{b}_j$ are constructed around a random vector chosen by the Haar measure. The distribution is perhaps surprisingly weighted towards large violation, with a mean value $\bar{S} \sim2.6$. In order to take into account experimental uncertainty $\delta$ in $S$ we can shift the local bound $\mathcal{L}$, modifying Bell-CHSH as
\begin{equation}
    |S| \le \mathcal{L} = 2 + \delta.
\end{equation}
Even with $\delta=0.2$, corresponding to only a few hundred detection events, the probability of violation for a perfect singlet state remains at $\sim 99.7\%$.

Of course, entangled states in prepared in the lab are never perfect. We use a partially mixed Werner state (\ref{eqn:werner-state}), whose purity is characterised by the \emph{visibility} $V$, to model this imperfection. Note that this is not equivalent to the visibility of quantum interference (\ref{eqn:hom-dip-visibility}).
Figure \ref{fig:chsh-distribution}(a, inset) shows the probability of violation as a function of $V$, demonstrating the robustness of generic nonlocality to imperfect experimental state preparation. For example, with $V=0.9$ and $\delta=0.1$, the probability of violation remains greater than $98.2\%$.

\subsection{Experiment} 

\begin{figure}[!t]
\centering
\includegraphics[width=\linewidth]{chapter5/fig/chsh/data_triads.pdf}
\caption[Bell tests using random measurement triads: experimental results]{ Bell tests requiring no shared reference frame. (a) 100 successive Bell tests. In each iteration, both Alice and Bob use a randomly-chosen measurement triad. For each iteration, the maximal \gls{chsh} value is plotted (black points). We observe \gls{chsh} violation in all trials; the red line indicates the local bound ($S=2$). The smallest \gls{chsh} value is $\sim 2.1$, while the mean \gls{chsh} value (dashed line) is $\sim 2.45$. This leads to an estimate of the visibility of $V=\frac{2.45}{2.6}\simeq 0.942$, to be compared with $0.913\pm0.004$ obtained by maximum likelihood quantum state tomography \cite{James2001}. This slight discrepancy is due to the fact that our entangled state is not exactly of the form of a Werner state. Error bars are too small to draw. (b) The experiment of (a) is repeated with reduced visibility of quantum interference, illustrating the robustness of the scheme. Each point shows the probability of \gls{chsh} violation estimated using 100 trials. Uncertainty in probability is estimated as the standard error.  Visibility for each point is estimated by state tomography, where the error bar is calculated using a Monte Carlo approach. Red points show data corrected for accidental coincidences. The black line shows the theoretical curve from Fig. \ref{fig:chsh-distribution} (inset). \label{fig:chsh-triads-expt}}
\end{figure}

The scheme described here is immediately applicable to a broad variety of scenarios, physical systems, and qubit encodings, including polarization states of entangled photons in optical fiber and free space, and path-encoding in photonic chips. We chose to perform our experimental implementation using the \textsc{cnot-mz} chip previously described, providing two path-encoded qubits with arbitrary state preparation and measurement capabilities. The scheme for reference-frame independent Bell violation described here is not absolutely necessary in order to violate \gls{chsh} on the \textsc{cnot-mz}, as alignment of reference frames is relatively straightforward. However, as we show in the next section (\ref{sec:random-chsh-no-calibration}), an extension to this scheme allows Bell violation with the \acrshort{cnotmz} in a ``black-box'' scenario, using completely uncalibrated phaseshifters.

We experimentally tested the situation in which Alice and Bob measure the singlet using orthogonal measurement triads. 
We prepare the singlet state using indistinguishable photons from a type-I SPDC source, together with the \textsc{cnot}' gate and local rotations as described in section \ref{sec:cnot-mz-tomography-experiment}. 
We then generate randomly chosen measurement triads $\vec{a}_i, \vec{b}_j$ using a pseudo-random number generator \cite{Mezzadri2006}.
Having calibrated the phase/voltage relationship of the phase shifters as described in section \ref{sec:cnot-mz-calibration}, we then apply appropriate voltages to phaseshifters 
$\phi_{5-8}$
in order to perform the nine measurements, evaluating $E_{ij}$.
For each measurement setting, two-photon coincidence counts between all 4 combinations of \glspl{apd} ($C_{00}$, $C_{01}$, $C_{10}$, $C_{11}$) are then measured for a fixed amount of time. The typical rate of simultaneous photon detection coincidences was $\sim1$ kHz. 
From this data we compute the maximal \gls{chsh} value as detailed above,  and the entire procedure is repeated 100 times. 
The results are presented in Fig.~\ref{fig:chsh-triads-expt}(a), where accidental coincidences, arising primarily from photons originating from different down-conversion events, which are measured throughout the experiment, have been subtracted from the data. 
Remarkably, all 100 trials lead to a clear \gls{chsh} violation; the average \gls{chsh} value we observe is $\sim 2.45$, while the smallest measured value is $\sim 2.10$. 

The visibility of the highest-fidelity experimental state was $0.913\pm0.004$, measured by maximum-likelihood quantum state tomography. 
Experimental imperfection in the photon source, \textsc{cnot-mz} device, and phaseshifter calibration all account for reduced visibility of the state, as described in \ref{sec:cnot-mz-errors}. 
In order to further test the robustness of the reference-frame-independent scheme described here, we deliberately introduced a temporal delay between the two photons at the SPDC source, increasing their distinguishability. 
The effect is as though the \textsc{cnot}' gate implements an incoherent mixture of the \textsc{cnot}' and identity operations \cite{OBrien2003}. 
Note that this does not reproduce the Werner state $\dema_V$, instead approximating
$ \dema =p \ket{\Psi^-}\bra{\Psi^-} + (1-p) \ket{01}\bra{01}$, where $p$ depends non-trivially on the temporal delay.

We repeat the protocol described above for a range of visibilities, estimating the visibility of the state through tomographic reconstruction of the experimental density matrix.  Figure \ref{fig:chsh-triads-expt}(b) clearly demonstrates the robustness of our scheme, in good agreement with theoretical predictions: a considerable amount of mixture must be introduced in order to significantly reduce the probability of obtaining a \gls{chsh} violation.  The discrepancy between experiment and theory is largely due to tomographic errors and the fact that we do not exactly prepare the Werner state (\ref{eqn:werner-state}).  Together these results show that large Bell violations can be obtained without a shared reference frame, even with realistic experimental imperfections. 

\section{Bell Tests without calibrated devices} 
\label{sec:random-chsh-no-calibration}
%\subsection{Theory}

Although Alice and Bob do not need to share a reference frame in order to implement the scheme described above, they nonetheless require \emph{well-calibrated} measurement devices in order to construct mutually unbiased measurement triads. Calibration of measurement devices, such as wave-plates, phaseshifters, etc. is a routine task, but may be challenging or even impossible in certain scenarios, forcing Alice and Bob to measure in completely random, non-orthogonal bases, which are unlikely to be uniformly distributed on the 2-sphere.  

It was shown in \cite{Liang2010} that if Alice and Bob choose measurements entirely at random, the probability of violation is $p \sim 28\%$. If they make $n$ repeated measurements of $S$ using random settings, they will asymptotically approach unit probability of eventual violation as $P_n \sim 0.72^n$. Can they do better than this?

\subsection{Theory} 
Assume that Alice and Bob measure all possible expectation values $E_{ij}$ over $m$ random measurement settings per party
$\vec{a}_i$ and $\vec{b}_j$. We can again write these expectation values in matrix form,
\begin{equation}
\mathcal{E}=
\left(
\begin{array}{ccccc}
E_{00} & E_{01} & E_{02} & E_{03} & \ldots\\
E_{10} & E_{11} & E_{12} & E_{13} & \ldots\\
E_{20} & E_{21} & E_{22} & E_{23} & \ldots\\
E_{30} & E_{31} & E_{32} & E_{33} & \ldots\\
\ldots & \ldots & \ldots & \ldots & \ddots 
\end{array}
\right)
\end{equation}
Now, there are an enormous number of ways in which groups of four expectation values from $\mathcal{E}$ can be combined to form valid \gls{chsh} inequalities in the form of (\ref{eqn:chsh-two}).
As a result, although it is no longer guaranteed that we will obtain nonlocal statistics, the probability of violation increases rapidly with $m$ to the extent that for $m=5$, assuming a perfect singlet state, $p\sim99.5\%$. This approach is similarly robust to limited visibility of the state, yielding $\sim97\%$ probability of violation when $V=0.9$ and $m=5$. Figure \ref{fig:chsh-distribution} shows the results of numerical simulations of this scenario for $m\in\left[2, 8\right]$.

\subsection{Experiment} 

\begin{figure}[!t]
\centering
\includegraphics[width=.9\linewidth]{chapter5/fig/chsh/data_voltages.pdf}
\caption[Bell tests using uncalibrated devices: experimental results]{
\label{fig:chsh-random-voltages}  
Experimental Bell tests using uncalibrated devices. 
We perform Bell tests on a two-qubit Bell state using uncalibrated measurement interferometers, choosing voltages uniformly from the interval $\left[0, 7\right]V$. 
For $m=2,3,4,5$ local measurement settings, we perform 100 trials (for each value of $m$). 
As the number of measurement settings $m$ increases, the probability of obtaining a Bell violation rapidly approaches one. 
For $m\geq 3$, the average \gls{chsh} value (dashed line) is above the local bound of CHSH=2 (red line). 
Error bars were estimated by a Monte Carlo technique, assuming Poissonian statistics. 
This data has been corrected for accidental coincidences.}
\end{figure}

Although the phaseshifters in the \textsc{cnot-mz} device had been well calibrated prior to this experiment, we emphasise the time-consuming nature of the calibration procedure, and the fact that the phase-voltage relationship is not consistent across heaters. To further complicate calibration, the phase-voltage response of an individual heater will drift with use over time.  In order to demonstrate the robustness of the above scheme to non-uniform randomness in the choice of measurement settings, we performed Bell-CHSH tests \emph{without }making use of the available phase-voltage information for phaseshifters $\phi_{5-8}$.

Having prepared the singlet state using the $\textsc{cnot}'$ gate, we chose the measurement operators 
$\vec{a}_i$ and $\vec{b}_j$ by randomly picking voltages in the interval $\left[0, 7\right]V$ for phaseshifters $\phi_{5, 6}$ and $\phi_{7, 8}$ respectively, where 7V is simply a hardware limitation of the heaters.
Since the phase-voltage response of each heater is nonlinear (see section \ref{sec:cnot-mz-calibration}), 
this gives rise to phases which are not uniformly distributed in the interval $\left[0, 2\pi\right]$, and therefore measurement bases $\hat{A}_i$, $\hat{B}_j$ which are certainly not chosen by the Haar measure.

We implemented this protocol for $m \in \{2,3,4,5\}$, observing a rapid increase in the probability of violation with $m$, as shown in figure \ref{fig:chsh-random-voltages}.  For $m=5$, we find 95 out of 100 trials lead to a \gls{chsh} violation, even when the choice of measurement is not uniformly random.  The visibility $V$ of the state used for this experiment was measured using state tomography to be $0.869\pm0.003$.

\section{Discussion} 
Often, entanglement and nonlocality are seen as rare and fragile phenomena, extremely sensitive to experimental noise and imperfection. By showing that nonlocality can be robustly detected without the need to calibrate or align measurement devices, even with limited visibility of state preparation, we have provided a new fundamental insight into the generic nature of nonlocality.

The schemes described here potentially have practical applications. First, Bell tests provide an unambiguous and device-independent test for the presence of entanglement --- a powerful tool for the future development of quantum technologies --- and the ability to perform such tests without calibration or alignment will likely facilitate such tests in some scenarios. The necessary criteria for a loophole-free reference-frame independent Bell test using the scheme described in section \ref{sec:random-chsh-calibrated} are discussed in further detail by G\'omez et al. \cite{Gomez2013}, paying particular attention to detection efficiencies. A further experimental implementation of Bell tests using orthogonal triads has been performed by Matthew Palsson et al. \cite{Palsson2012}. It has recently been proposed \cite{Brask2013} that this scheme might also facilitate the detection of nonlocality of a \emph{single} photon (see, for example, ref. \cite{Tan1991a}).

We also expect that this work will find applications in quantum communication protocols.
Previous work by Laing et al. \cite{Laing2010} describes a technique for reference-frame independent QKD in which the parties share in advance a single common measurement basis, and this theory has recently been implemented in collaboration with Nokia \cite{Zhang2013, Wabnig2013}. Our work extends this capability to the case where no knowledge of the reference-frame is shared.  More recent interest in the general topic of alignment-free quantum communication has been reviewed by D'Ambrosio et al. \cite{DAmbrosio2012}, and an experimental implementation of device-independent QKD without a shared reference frame, using our theoretical framework, was recently described in \cite{Slater2013}.

%%%%%%%%%%%%%%%%%%%%%%%%%%%%%%%%%%%%%%%%%%%%%%%%%%%%%%%%%%%%%%%%%%%%%%%%%%%%%%%%%%%%%%%%%%%%%%
\section{A noise-powered entanglement detector}
\label{sec:noisy-entanglement-witness}
% define a few useful bits of notation
\newcommand{\astate}{\hat{\rho}_C}
\newcommand{\bstate}{\hat{\rho}_{AB}}
\newcommand{\ameas}{\hat{A}}
\newcommand{\bmeas}{\hat{B}}

%%%%%%%%%%%%% FIGURE 1 %%%%%%%%%%%%%
\begin{figure}[t!]
\centering
\includegraphics[width=.9\linewidth]{chapter5/fig/harsh/setup.pdf}
\caption[A noise-powered entanglement detector]{(a) Charlie has an untrusted source of two-qubit states $\astate$, which he claims is entangled. Alice and Bob want to reliably estimate some measure of entanglement generated at the source, but their view is obscured by an unstable unitary channel.  
(b) Photonic experimental implementation. Charlie has a type-2 SPDC source of photon pairs, which can be switched between entangled and separable operation. He sends photon pairs to Alice and Bob through non-polarization-maintaining optical fiber, which is continuously moved, bent, and twisted throughout the experiment.
(c) Numerical simulation, showing the distribution of $T=\frac{1}{N}\sum_i |\expect{ZZ}_i| $, when Charlie prepares any separable state (red lines) vs. any maximally entangled state (blue lines). Alice and Bob are thus able to distinguish entangled and separable sources. 
\label{fig:harsh-noise-setup}
}
\end{figure}
%%%%%%%%%%%%% END FIGURE 1 %%%%%%%%%%%%%


In the previous discussion, although the relative orientation of Alice and Bob's frames is unknown, it was assumed that this orientation does not change between consecutive measurements, i.e. every element of (\ref{eqn:chsh-three-matrix}) can be estimated before $\vec{a}_i, \vec{b}_j$ change by any appreciable amount.  However, for realistic unstable reference frames --- including polarization in fiber and bulk path interferometers --- the rate of change is often so high that this assumption does not hold.  

We now consider the situation in which each reference frame changes, uniformly and at random, every time Alice and Bob measure an expectation value. Na\"ively, it might appear that there is then very little that Alice and Bob can say about the state, as they are forced to make observations through a ``fog'' of random, uncorrelated local unitary rotations. However, we will introduce a simple protocol which \emph{exploits} this noise, allowing Alice and Bob to distinguish between entangled and separable sources, and estimate certain physical properties of the state. We discuss immediate practical applications of this scheme with respect to state characterization and secure communication.

\newcommand{\fullchannel}{\hat{\mathcal{U}}}
The experimental scenario is illustrated in figure \ref{fig:harsh-noise-setup}.
Charlie has a source, which generates qubit pairs in the state $\astate$. He claims that $\astate$ is entangled, but this claim is not trusted. Charlie sends qubit pairs to two observers, Alice and Bob, who measure their respective systems in a local basis before comparing measurement outcomes. The channels between Charlie and Alice/Bob, corresponding to unitary operators $\unitary_A(t)$, $\unitary_B(t)$, are assumed to be lossless but unstable. At some time $t$, the two-qubit channel $\fullchannel$ is described by a unitary operator 
\begin{equation}
    \fullchannel(t) = \unitary_A(t) \otimes \unitary_B(t),
\end{equation}
where $\unitary_A(t)$, $\unitary_B(t)$ are chosen independently and at random from the Haar measure on $SU(2)$. After some interval $\Delta t$, instability in the channel leads to new instances of $\unitary_A$, $\unitary_B$, drawn again from the Haar measure. During a single timestep $t_j = t_0 + j \Delta t$, Alice and Bob receive $n$ copies of the state
\begin{equation}
   \bstate^i = \fullchannel(t_j) ~ \astate ~ \fullchannel(t_j)^\dagger = \fullchannel_j ~ \astate ~ \fullchannel_j^\dagger 
\end{equation}
where $n$ is sufficiently large to give a good estimate of the expectation value
\begin{equation}
  E_j = \expect{\ameas_j \otimes \bmeas_j } = \text{Tr}\, ( \bstate^j ~ \ameas_j \otimes \bmeas_j ),
\end{equation}
where $\ameas$, $\bmeas$ are Alice and Bob's single-qubit measurement operators  respectively.

Alice and Bob would now like to determine whether Charlie's state is entangled. Note that Charlie is not held accountable for the behaviour of the channel --- Alice and Bob care about the degree of entanglement of $\astate$, which is independent of $\fullchannel$.
Averaging over all time ($t \rightarrow \infty$),  the state seen by Alice and Bob is 
\begin{equation}
    \bstate^\infty = 
   \int_0^\infty \text{d}t ~ (\fullchannel(t) ~\astate ~\fullchannel(t) ^\dagger) =
   \int_{SU(2)\otimes SU(2)} \text{d} \, \fullchannel ~ (\fullchannel \, \astate \, \fullchannel ^\dagger).
\end{equation}
Since the defining representation of $SU(2)\otimes SU(2)$ is irreducible (all local two qubit operations leave no nontrivial subspaces invariant), Schur's lemma implies that $\bstate^\infty$ is proportional to the identity regardless of $\astate$, and due to normalization, $\bstate^\infty=\mathbf{1}/4$, i.e. Alice and Bob sees a maximally mixed state. 

What happens if Alice and Bob attempt to perform quantum state tomography (section \ref{sec:state-tomography}), ignoring fluctuations in the channel? Since tomography depends on a finite number of measured expectation values ($t<\infty$), the reconstructed density matrix is not necessarily maximally mixed, but nevertheless provides an unfaithful representation of $\astate$, and is not guaranteed to contain information on the degree of entanglement. If Alice and Bob attempt to evaluate CHSH, the situation is even worse: a basic condition for \gls{chsh} is that the state should not change between measurements, and when this condition is broken Alice and Bob can erroneously detect a Bell violation even when Charlie's state is separable. In fact, numerical simulations indicate that separable and entangled sources both violate \gls{chsh} with equal probability, $\sim1\%$. 
%Further analysis of the performance of tomography and \gls{chsh} under local random noise is given in Appendix \ref{app:noisy} of this thesis.

Assuming that $\unitary_A$, $\unitary_B$ are Haar-random, Alice and Bob know that no particular choice of local measurement basis can give more information than any other. Without loss of generality, we can therefore assume that they \emph{always} measure the $\hat{\sigma}_z$ basis, obtaining expectation values
\begin{equation}
    E_j = \langle \hat{\sigma}_z \otimes \hat{\sigma}_z \rangle = \text{Tr}\, \left( (\hat{\sigma}_z \otimes \hat{\sigma}_z ) (\fullchannel_j \astate \fullchannel_j^\dagger ) \right).
    \label{eqn:harsh-expectation-value}
\end{equation}
It can easily be shown that the average value of $E_j$ over all $\fullchannel_j$ is always 0, regardless of $\astate$. However, if we take the absolute value of $E_j$ before averaging, we will show that the quantity
\begin{equation}
    T \equiv \langle |E| \rangle = \sum_{i=0}^N \frac{|E_j|}{N},
    \label{eqn:define-t}
\end{equation}
distinguishes entangled states from separable states, and can be used to infer the degree of entanglement of $\astate$.  Figure \ref{fig:harsh-noise-setup}(c) is result of a numerical simulation, showing the distribution of $T$ for a separable state $\ket{00}$ and a maximally entangled state $\ket{\Psi^-}$, averaging over $N$ measurements. 
All separable pure states give a mean value $T=1/4$, while all maximally entangled two-qubit states give $T=1/2$. Partially entangled states give intermediate values. Maximally mixed states give $T=0$.  As the number of averages $N$ is increased, each probability distribution converges towards a Gaussian profile with FWHM proportional to $1/\sqrt{N}$, following the central limit theorem. By taking an increasing number of measurements, Alice and Bob can therefore distinguish a separable state from an entangled state to an arbitrary confidence level.

\vspace{3mm}
\noindent\textbf{Proof:}
Given that $\fullchannel = \unitary_A \otimes \unitary_B$ where $\unitary_{A,B}$ are chosen by the Haar measure on $SU(2)$, we can assume without loss of generality that, after the channel, any separable state is equivalent to the state $\ket{00}$ and any maximally entangled state is equivalent to the singlet $\ket{\Psi^-}$. Unitary rotation of a qubit followed by measurement in the $\hat{\sigma}_z$ basis is equivalent to measurement in the \emph{effective} basis $\hat{M}^{eff} = \unitary^\dagger \hat{\sigma}_z \unitary$. Rather than integrating $\unitary_{A,B}$ over the Haar measure on $SU(2)$, it is simpler to consider the Bloch vectors $\vec{a}_j, \vec{b}_j \in \mathbb{R}^{(3)}$, which depend on $\unitary_{A,B}$ and map to Alice and Bob's effective measurement operators 
\begin{align}
\hat{A}_j^\text{eff} = \vec{a}_j \cdot \vec{\sigma} = a_j^x \hat{\sigma}_x + a_j^y \hat{\sigma}_y + a_j^z \hat{\sigma}_z\\
\hat{B}_j^\text{eff} = \vec{b}_j \cdot \vec{\sigma} = b_j^x \hat{\sigma}_x + b_j^y \hat{\sigma}_y + b_j^z \hat{\sigma}_z,
\end{align}
which correspond to points on the 2-sphere $\mathbb{S}^{(2)}$. For the singlet, the expectation value (\ref{eqn:harsh-expectation-value}) is simply given by the dot product, $E_j=-\vec{a}_j \cdot \vec{b}_j$. For $\ket{00}$, the expectation value is $E_j = a_j^z\, a_j^z$.



In order to compute the average value of $T$ in the asymptotic limit of infinite statistics, we must now integrate $|E|$ over $SU(2)\times SU(2)$. This is equivalent to integrating each Bloch vector over the 2-sphere,
\begin{equation}
    \langle |E| \rangle_\infty = \int_{\mathbb{S}^{(2)}} d \vec{a}  \int_{\mathbb{S}^{(2)}} d \vec{b}  ~ |E|.
    \label{eqn:harsh-integral}
\end{equation}
First, consider the singlet state. The absolute expectation value $|E| = |\vec{a} \cdot \vec{b}| = |\cos(\phi)|$ depends only on the angle $\phi$ between $\vec{a}$ and $\vec{b}$. To emphasise, it depends only on the \emph{relationship} between the two channel unitaries. Without loss of generality, we can therefore fix $\vec{a}$ such that $\hat{A}_1 = \hat{\sigma}_z$. Then, we integrate $|E|$ over $\vec{b}$ using a single parameter, $\phi$, which rotates $\vec{b}$ about the $x$-axis of the Bloch sphere. In order to integrate this angle uniformly over $\mathbb{S}^{(2)}$, we must take $\phi = \cos^{-1}(2v-1)$, where $v$ is uniformly distributed in the interval $\left[0, 1 \right]$. Now,
\begin{equation}
    \langle |E| \rangle_\infty = \int_{0}^{2\pi} d \phi |\cos \phi| = \int_{0}^{1} dv |2v-1| = \frac{1}{2}.
\end{equation}
Now consider the separable state. The absolute expectation value $|E| = |a^{z} \cdot b^{z}|$ depends on both the relationship \emph{and} the individual directions of $\vec{a}$, $\vec{b}$. Writing this expression in terms of the angles $\phi_1$, $\phi_2$ between $\hat{z}$ and $\vec{a}$, $\vec{b}$ respectively, we have $|E| = |\cos \phi_1 \cos \phi_2|$. Using the same parametrization to uniformly integrate over $\mathbb{S}^{(2)}$ in (\ref{eqn:harsh-integral}), this becomes
\begin{eqnarray}
   \langle |E| \rangle_\infty &=& \int_{0}^{2\pi} d \phi_1 \int_{0}^{2\pi} d \phi_2 |\cos \phi_1 \cdot \cos \phi_2|\\  
             &=& \int_{0}^{1} dv_1 \int_{0}^{1} dv_2 |(2v-1)(2v-1)| = \frac{1}{4}.
\end{eqnarray}
So, all maximally entangled two-qubit states have $\bar{T}=1/2$ and all separable states have $\bar{T}=1/4$.
$\hfill\blacksquare$
\vspace{5mm}

%%%%%%%%%%%%% Heatmap figure  %%%%%%%%%%%%%
\begin{figure}[t!]
\centering
\includegraphics[width=\linewidth]{chapter5/fig/harsh/state_numerics.pdf}
\caption[Detecting entanglement in mixed and partially entangled states]{
(a) Numerical simulation of average $T$ values for states of varying purity and concurrence. The heatmap shows the mean value of $T$ as a function of the purity 
$Tr(\astate^2)$
and concurrence 
$\mathcal{C}(\astate)$
, over states parametrized as 
$\astate(v, \mu) = v \ket{\phi(\mu)} \bra{\phi(\mu)} + (1-v)\mathbf{1}/4$
, with 
$\ket{\phi(\mu)} = \sqrt{\mu}\ket{\Psi^-} + \sqrt{1-\mu}\ket{00}$.
The white area of the figure is unphysical:  maximally mixed states cannot be maximally entangled. Wavelike features in the figure are an artefact of the numerical interpolation method. (b) Numerical simulation, showing the behaviour of $\expect{|E|}$ when the channel fluctuates on a timescale shorter than that required to measure a single expectation value.
\label{fig:harsh-noisy-states}
}
\end{figure}
%%%%%%%%%%%%% Heatmap figure %%%%%%%%%%%%%


To see how this scheme performs for states other than $\ket{00}$ and $\ket{\Psi^-}$, we consider the state
\begin{equation}
    \astate(v, \mu) = v \ket{\phi(\mu)} \bra{\phi(\mu)} + (1-v)\mathbf{1}/4,
\end{equation}
 where 
$\ket{\phi(\mu)} = \sqrt{\mu}\ket{\Psi^-} + \sqrt{1-\mu}\ket{00}$,  which can be continuously tuned between a maximally entangled pure state, a separable pure state, and the maximally mixed state. Figure \ref{fig:harsh-noisy-states}(a) shows the results of a numerical calculation of the mean value of $T$, as a function of the purity and concurrence of $\astate(v, \mu)$ for various values of $v$ and $\mu$. As we would expect of a sensible entanglement measure, $T$ is maximal for a maximally entangled state ($T=\frac{1}{2}$) and minimal for the maximally mixed state ($T=0$). As the concurrence or purity of the state is reduced, the strength of correlations is naturally reduced and $T$ falls off monotonically. Note that for separable states, $T$ also gives a measure of purity.

\subsection{Experiment}
%%%%%%%%%%%%% FIGURE 2 %%%%%%%%%%%%%
\begin{figure}[t!]
\includegraphics[width=.99\linewidth]{chapter5/fig/harsh/data.pdf}
\caption[Detecting entanglement by bending single-mode fiber]{Experimental data. (a) Expectation values $E_j = (C_{00}-C_{01}-C_{10})/\mathcal{C}$, measured as a function of time for a Bell state (blue lines) and a separable state (red lines), with optical fiber subject to constant bending and twisting. (b) Values of $T$ computed from the data shown in (a), with $N \in \left[1, 20\right]$. The entangled distributions (red lines) are clearly distinguishable from the separable state data (blue lines). By encoding bits of information in the choice of entangled/separable state, an image can be sent through the noisy polarization channel. Inset (i) source image sent by Charlie, (ii) image recovered by Alice and Bob. (c) Twisting optical fiber does not sample uniformly from $SU(2)$. This data was measured using waveplates to experimentally implement $\sim100$ unitaries sampled numerically from $SU(2)\times SU(2)$. Blue and red dots show the quantum and classical experimental distributions of $T$ respectively, for $N=4$. Solid lines show the theoretical prediction. Inset: real and imaginary parts of $\ket{\Psi^-}$ as generated by the source, characterized by quantum state tomography.
}
\label{fig:harsh-noise-data}
\end{figure}
%%%%%%%%%%%%% END FIGURE 2 %%%%%%%%%%%%%

%In order to experimentally test this scheme we used polarization-entangled photon pairs generated by SPDC to encode two qubits. each of which was subject either to Haar-random rotations, or random rotations due to mechanically stressed \acrshort{smf}. 
We experimentally tested this scheme using polarization-entangled photon pairs, with both artificial and environmental sources of instability.

\subsubsection{Experimental setup} 
The experimental setup is shown in figure \ref{fig:harsh-noise-setup}.
We used a type-II spontaneous parametric downconversion source, as described in section \ref{sec:spontaneous-parametric-downconversion} of this thesis, to generate entangled photon pairs at \SI{808}{\nano \metre}. A \SI{404}{\nano \metre} Toptica \emph{iBeam} laser at \SI{60}{\milli \watt} was focussed to a waist of \SI{\sim 40}{\micro \metre} on a \SI{2}{mm}-thick \acrshort{bibo} crystal. We collected down-converted photons at the intersection of the two cones as shown in figure \ref{fig:spdc_schematic}, using two prisms. Each photon was sent through an arrangement of quarter-wave and half-wave plates, allowing arbitrary $SU(2)$ polarization rotations to be applied. A \SI{1}{\milli \metre}-thick uniaxial \acrshort{bibo} crystal was used to compensate for temporal walk-off between horizontal and vertical polarizations. Each arm of the source was then coupled into \SI{\sim 4}{\metre} of \acrshort{smf} (OZ optics, \SI{808}{\nano \metre}). The measurement setup consisted of two fibre-coupled \gls{pbs} and four Perkin Elmer \gls{apd} single-photon detectors, and allows polarization readout in the $\{\ket{H}, \ket{V}\}$ basis. A linear polarizer was optionally inserted into each arm, before the fiber, allowing projective measurements to be performed without the influence of uncontrolled polarization rotations due to the fiber.
%We performed experiments using two different sources of noise: artificial, uniform Haar-random rotations implemented using waveplates, and  biased rotations due to stress-induced birefringence of single-mode fiber.

\subsubsection{Source characterization} 
The source was optimized to prepare the Bell state 
\begin{equation}
   \ket{\Psi^+} = \frac{1}{\sqrt{2}}\left(\ket{01}+\ket{10}\right) =
\frac{1}{\sqrt{2}}\left(\ket{HV}+\ket{VH}\right),
\label{eqn:spdc-bell}
\end{equation} 
where the phase between $\ket{HV}$ and $\ket{VH}$ terms is determined by the orientation of the compensation crystals. The experimental state was characterized by full quantum state tomography. The waveplates and polarizers shown in figure \ref{fig:harsh-noise-setup} were used to implement 36 linearly independent, mutually unbiased measurements, as described for path encoding in section \ref{sec:state-tomography}, and the state was then reconstructed using the same standard maximum-likelihood technique. Real and imaginary parts of the reconstructed density matrix are shown in figure \ref{fig:harsh-noise-data}(c, inset). The quantum state fidelity with respect to $\ket{\Psi^-}$ was found to be $0.965 \pm 0.002$.  After losses due to optical elements, fiber coupling, and detector inefficiency, the twofold count-rate registered at the detectors was typically $\sim 1000$ counts per second.

\subsubsection{Environmental noise} 
As already discussed, the polarization of light is not maintained by \acrshort{smf}. The birefringence of the fiber is affected by mechanical stress, and a piece of uniformly stressed fiber has an equivalent effect to a wave-plate, whose characteristic phaseshift depends on the strain, fiber length, core diameter, temperature, and wavelength of light. In fact, \emph{arbitrary} $SU(2)$ polarization rotations can be accomplished using a single piece of \acrshort{smf}, by applying controlled stress to three different regions --- these devices are typically marketed as ``fiber polarization controllers''. A section of fiber exposed to uncontrolled temperature variation and mechanical vibration in the ambient environment of the laboratory will therefore tend to effect a slowly time-varying, arbitrary, random polarization rotation upon the light it carries. The fact that telecom optical fiber networks do not typically use polarization encoding is partly due to cost involved in overcoming this effect.

In our first experiment, we investigated the performance of our technique using random unitary rotations generated in this way. Removing both polarizers, we connected each arm of the source to a fibre-coupled \gls{pbs} which, together with two detectors, projects onto the $\ket{H}$, $\ket{V}$ states --- i.e. measurement in the $\pauli_z$ basis. We then recorded coincidence count-rates $c_{HH}, c_{HV}, c_{VH}, c_{VV}$, corresponding to the $\ket{HH}, \ket{HV}, \ket{VH}, \ket{VV}$ basis states respectively, while manually straining, bending and shaking both optical fibers. We accumulated $\sim 250$ expectation values
\begin{equation}
    E_j = \frac{c_{HH} - c_{HV} - c_{VH} + c_{VV}}{c_{HH} + c_{HV} + c_{VH} + c_{VV}}.
\end{equation}
Inserting a linear polarizer at \SI{0}{\degree} into one arm of the source, we then filtered out the $\ket{HV}$ term of (\ref{eqn:spdc-bell}) --- rendering Charlie's state separable --- and took a second set of data, manipulating the fibers as before. Raw data for both states is shown in figure \ref{fig:harsh-noise-data}. Although the average expectation value $\expect{E_j}$ is equal to zero for both states, the entangled state is clearly more likely to yield strongly correlated or anticorrelated statistics. Figure \ref{fig:harsh-noise-data} shows the distribution of the absolute expectation value, when averaging over $N$ measurements, for $N \in \left[ 1, 20 \right]$. As predicted, we see a clear distinction between distributions generated by the entangled and separable states, becoming increasingly pronounced with larger values of $N$. When Charlie prepares $\ket{\Psi^+}$, the average value of $T$ (\ref{eqn:define-t}) was found to be $\expect{T}\sim 0.399$. For the separable state $\ket{VH}$, we found $\expect{T}\sim0.163$. 

The distribution of random unitaries generated by manual manipulation of \acrshort{smf} is not perfectly uniform. Moreover, it is inevitable that the fiber will move somewhat during each measurement step, in which case the measured expectation value is averaged over a continuum of states. This leads to an overall reduction in the absolute expectation value, and explains why measured values of $\expect{T}$ were not closer to 1/2 and 1/4 respectively. Experimentally, it will often be the case that the channel unitary will change by a significant amount during the measurement of a single expectation value. A numerical analysis of the performance of this scheme under such conditions is shown in figure \ref{fig:harsh-noisy-states}(b). Although $\expect{T}$ does indeed reduce as the maximum rate of change of the channel is increased, we note a consistent separation between entangled and separable states, suggesting that they might still be distinguished even when the channel fluctuates much faster than a measurement can be made.

In order to illustrate a possible practical application of this scheme, we consider a situation in which Charlie must send a message to Alice and Bob. By switching between entangled and separable state preparation, Charlie can encode the zero and one states of a classical bit, which can then be read out by Alice and Bob --- despite noise on the channel. We used this approach to send 100 bits of data, comprising an image of the character $\pi$, from source to observer with a statistical fidelity of \SI{85}{\percent}. Results are shown in figure \ref{fig:harsh-noise-data}(b, inset).

\subsubsection{Haar-random noise} 
Stressed optical fibre provides a practical example of an unstable environmental channel, but does not typically sample uniformly from $SU(2)$. In order to perform a more controlled test of the theoretical results outlined above, we took measurements using an arrangement of waveplates to implement each qubit channel. Polarizers were inserted into each beam, allowing each qubit to be projected into the $\{ \ket{H}, \ket{V} \}$ basis without any influence from the fibre. By setting the fast-axis angles of three consecutive waveplates (\gls{qwp}, \gls{hwp}, \gls{qwp} in that order), any unitary polarization rotation in $SU(2)$ can be realized.  Following the approach of Mezzadri \cite{Mezzadri2006}, we sampled $\sim 40$ pseudo-random separable two-qubit unitaries from the Haar measure and solved for the requisite waveplate angles. Setting these angles to Alice and Bob's waveplates, we measured expectation values for both the separable state and the singlet, as before. The distribution of experimentally measured expectation values is shown, for each state, in figure \ref{fig:harsh-noise-data}(c). We measured mean values of $\expect{T}$ of $0.5 \pm 0.1$ and $0.18 \pm 0.07$ for the entangled and separable state respectively, compared to ideal theoretical values of $0.5$ and $0.25$.

\section{Discussion}
The results presented here allow us to formalize a commonly-held, natural and accessible notion of entanglement. Taking two separate systems, a random local operation is applied to each system. Each system is then measured in a local basis. Our main result simply formalizes the fact that, on average, entangled systems yield more strongly correlated measurement outcomes than separable systems. Since our scheme is completely reference-frame independent, we can give this description without speaking of any explicit choice of measurement operators, waveplate angles, or even specific states. 

A compelling property of this scheme is the beneficial function of Haar-random, or ``white'' noise. We see the greatest statistical separation between entangled and separable states, and thus obtain the most information, when instability in the channel is Haar-random. Consider the situation illustrated in figure \ref{fig:apps}. Alice and Bob must assess the degree of entanglement of Charlie's state. They are forced to receive qubits from Charlie over an unstable channel, which is untrusted but guaranteed to be local (i.e. $\fullchannel \in SU(2) \otimes SU(2)$). In this scenario, there is no guarantee that the channel is Haar-random --- it may even be the case that Charlie is deliberately manipulating the channel. However, Alice and Bob can effectively ``cancel out'' any local operations that may occur on the channel, by \emph{deliberately measuring in a Haar-random basis}. It is then vanishingly unlikely that Alice and Bob will register a large value of $T$ (i.e. $T\sim0.5$), unless Bob truly has access to an entangled source, or is able to learn Alice and Bob's choice of measurement setting. This ability to override unknown noise on a channel using controlled ``white noise'', while still obtaining meaningful information on the source, has obvious practical implications for the characterization of quantum states and processes. It would be interesting to consider possible applications outside photonics, where an entangled state must be observed through a noisy local channel.

%%%%%%%%%%%%% APP FIGURE %%%%%%%%%%%%%
\begin{figure}[t!]
\centering
\includegraphics[width=.8\linewidth]{chapter5/fig/harsh/apps.pdf}
\caption[Potential applications]{
Alice and Bob are tasked with assessing the entangling capability of Charlie's apparatus, but are forced to communicate with him over an untrusted, noisy local channel. By measuring in a controlled Haar-random local basis, they effectively override any fluctuation in the channel, and are able to reliably confirm or deny that Charlie's state is entangled. Even if Charlie has control of the channel, he would need to gain information on Alice and Bob's measurement settings in order to cheat in this scenario.
}
\label{fig:apps}
\end{figure}


\section*{Statement of work}
All of the experimental data presented here was measured by myself, except for the density matrix in section \ref{sec:noisy-entanglement-witness}. I conceived the original idea in  section \ref{sec:noisy-entanglement-witness}. The proof of measurement triads, and figure \ref{fig:chsh-distribution} are due to my co-authors.

\bibliographystyle{unsrt}
\bibliography{main.bib}

