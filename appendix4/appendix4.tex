\chapter{Metadata}
\label{app:metadata}

\begin{figure}[t]
\begin{center}
\includegraphics[width=\linewidth]{progress.pdf}
\end{center}
\caption[Metadata]{Writing a PhD thesis. }
\label{fig:metadata}
\end{figure}

In writing my thesis, the PhD theses of my colleagues and forbears (notably Jonathan Matthews, Alberto Politi, Alberto Peruzzo, Damien Bonneau, Dylan Saunders and Nathan Langford) have been an indispensable source of detailed, relevant information, clear explanation, and a model for the style and structure of a thesis. 

In the hope that it might be useful to other PhD students going through the same process, I include the dataset shown in figure \ref{fig:metadata}. This is a log of approximate word count of my thesis, recorded every time I committed a revision to my \texttt{git} repository. Three aspects of this figure are interesting: first, the striking linearity of the curve, which I absolutely expected to be a polynomial with positive second derivative. Second, one can easily identify regions of ``burnout'' directly after large streaks of progress: I would suggest that this stop-start mode of operation be avoided as far as possible. The last observation, I will leave as an exercise for the reader.


