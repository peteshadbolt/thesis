\newpage
~\\
\vspace{50pt}
\begin{center}
{\Large \sc Abstract \\ \rm}
\vspace{25pt}
\end{center}

Quantum mechanics predicts phenomena which have no classical analogue. This modifies our understanding of the capability of physical machines. Single photons, together with simple interferometers and single photon detection, have been shown to be universal for the construction of many such machines. The nascent field of integrated quantum photonics addresses the scalability and practicality of such machines, and their integration in miniaturized monolithic chips.

In this work, we explore the scope and flexibility afforded by integrated quantum photonics, both in terms of practical problem-solving, and for the pursuit of fundamental science. We demonstrate and fully characterize a two-qubit quantum photonic chip, capable of arbitrary two-qubit state preparation. We make use of the unprecedented degree of reconfigurablility afforded by this device to implement a novel variation on Wheeler's delayed choice experiment, and test a new technique to obtain nonlocal statistics without a shared reference frame. We demonstrate a new algorithm for quantum chemistry, simulating the helium hydride ion. Finally, we demonstrate multiphoton quantum interference in a large Hilbert space, and discuss implications for computational complexity.


