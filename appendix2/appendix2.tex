\newcommand{\qy}{\texttt{qy}\xspace}
\chapter{\qy}
%\addtocontents{toc}{\protect\addvspace{10pt}}
%\addtocontents{toc}{Appendix 1}
\label{app:qy}

In the course of the experimental work described in this thesis we have developed a broad general-purpose base of computer code (\qy), which is maintained and documented as a library to encourage re-use. The majority of this code is written in the \texttt{Python} programming language\footnote{\texttt{http://www.python.org/}}, with some compiled extensions written in \texttt{C} or \texttt{Cython} for speed. Both of these languages are free and open source.

\qy includes modules for data acquisition (DAQ) and hardware control, data logging and analysis, and numerical simulation. The code is currently open source, and can be obtained via \texttt{git}:
\begin{center}
\framebox{\centering \texttt{https://github.com/peteshadbolt/qy}}\par
\end{center}
The top-level structure of the library is as follows:
\begin{itemize}
    \item \texttt{qy.analysis}: Various standard metrics and tools for data analysis.
    \item \texttt{qy.formats}: File formats, in particular an efficient format to represent multiphoton coincidence-counting data.
    \item \texttt{qy.graphics}: Utility functions for graphics and plotting.
    \item \texttt{qy.hardware}: Interfaces to various pieces of standard laboratory apparatus, including FPGA counting systems, the DPC-230 described in section \ref{sec:counting}, Toptica diode lasers, Coherent Ti:Saph lasers, custom powermeters, Thor labs SMC100 power meters, silica-on-silicon thermal phase shifters, etc.
    \item \texttt{qy.settings}: Utility functions to read, write and persist global settings.
    \item \texttt{qy.simulation}: Provides general quantum information primitives, including single-qubit states and operators, frequently used two-qubit states and gate operations, measures such as quantum state fidelity and concurrence, a circuit-model simulator, and an optimized linear-optics simulator, capable of calculating multiphoton states and statistics in arbitrary linear optical networks.
    \item \texttt{qy.util}: Utility functions.
    \item \texttt{qy.wx}: Extends the functionality of the \texttt{wx} GUI library.
\end{itemize}
Here we will discuss two components in particular: the \texttt{linear\_optics} simulation package and the \texttt{.counted} file format.

\section{Universal linear optics simulator}
\begin{figure}[t!]
\includegraphics[width=1\linewidth]{appendix2/benchmark.pdf}
\caption[Permanent routines: benchmarking]{
    Estimated performance of various implementations of algorithms to compute the permanent, tested against 1000 Haar-random $N \times N$ matrices. Red and blue lines show average execution times for for Ryser's algorithm, implemented in Python and Cython respectively, as a function of $N$. Green and black lines correspond to execution times for hard-coded implementations up to $N$=4, again in Python and Cython respectively.
    \label{fig:permanent-benchmark}
}
\label{fig:permanent-benchmarking}
\end{figure}
The module \texttt{qy.simulation.linear\_optics} provides a simple means to simulate multiphoton states and statistics in arbitrary linear optical circuits. This work draws upon ideas and code kindly provided by Jasmin Meinecke, Nick Russell, Jacques Carolan. The numerical method is exactly that described in section \ref{sec:permanents}, and as such depends almost entirely on the calculation of permanents. We have developed optimized code to compute the permanent of complex matrices, using a number of different algorithms and implementations. We implemented the core algorithm using \texttt{Cython}, a compiled language which can typically achieve much better performance than standard \texttt{Python}, which is interpreted rather than compiled. Typical real-world performance of these methods is summarized in figure \ref{fig:permanent-benchmark}.  The library is very easy to use:
\lstset{ language=Python, basicstyle={\footnotesize \tt}, showstringspaces=false, tabsize=1, xleftmargin=17pt, framexleftmargin=17pt, framexrightmargin=5pt, framexbottommargin=10pt, framextopmargin=10pt}

\lstinputlisting{appendix2/qydemo.py}

When computing the permanent, it was noticed (by Nick Russell) that hard-coded routines can give a significant advantage in speed for small matrices, as the overhead associated with loops and conditional statements can be completely avoided. For completeness we include code up to $N=4$, beyond which the advantage with respect to Ryser's algorithm is negligible.
\lstset{ language=Python, basicstyle={\tiny \tt}, showstringspaces=false, tabsize=1, xleftmargin=17pt, framexleftmargin=17pt, framexrightmargin=5pt, framextopmargin=8pt, framexbottommargin=4pt}
\newpage
\lstinputlisting{appendix2/perms.py}
\newpage

\section{Data file format: \texttt{.counted}}
In order to efficiently store coincidence-counting data generated by the DPC-230, we designed a custom binary file format. These files, assigned the extension \texttt{.counted}, are structured in records of three 4-byte words. The first word denotes the type of data in the record, and the following two words encode that data, as follows:
\lstset{ basicstyle={\tiny \tt}, showstringspaces=false, tabsize=1, xleftmargin=0pt, framexleftmargin=0pt, framexrightmargin=5pt, framexbottommargin=10pt, framextopmargin=10pt}
\lstinputlisting{appendix2/counted.py}
Now that the counting system is a little more mature, this format should really be retired in favour of a less opaque standard.

\newpage
\section{\acrshort{cnotmz} API}
Much is made in the popular press of the potential impact and power of quantum computing, however the subject is still treated with a certain amount of trepidation, owing to the percieved difficutly of the field, creating barriers to entry for engineers and scientists from other disciplines. In an effort to make quantum computing somewhat more tangible, we built an open-access interface to the \gls{cnotmz}, accessible through a web browser. Users can run simulations of multiphoton experiments, either through a \gls{gui}, or using an \gls{http} JSON \gls{api}. Once granted permission, they can then acquire data from the lab in real-time.

For further detail, see
\begin{center}
\framebox{\centering \texttt{https://cnotmz.appspot.com}}\par
\end{center}

\begin{figure}[b!]
\centering
\includegraphics[width=.8\linewidth]{appendix2/qcloud.jpg}
\caption[Quantum computing in the cloud]{
    Accessible multiphoton simulation of the \gls{cnotmz}, running in a web browser.
}
\label{fig:qcloud}
\end{figure}

